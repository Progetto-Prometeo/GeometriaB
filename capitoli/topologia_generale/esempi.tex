\chapter{Esempi}



\section{Alcune varianti di topologie su $\R$}
\subsection{\textcolor{TopGener}{\textbf{Caratteristiche degli spazi compatti}}}



\begin{definition}
	D'ora in poi sia $\mathcal{B} = \left\{\left[a,b\right) \,\middle|\, a,b \in \R\right\}$ una base tale da generare la topologia $\tau_j$ sull'insieme $\R$.
\end{definition}

\begin{theorem}
	La topologia $\tau_j$ generata dalla base $\mathcal{B} \coloneqq \left\{ \left[a,b\right) \,\middle|\, a,b\in \R \right\}$ è strettamente più fine della topologia euclidea. 
\end{theorem}
\begin{proof}
	Ogni aperto di $\tau_{\text{euclidea}}$ è contenuto in $\tau_j$. Infatti sia $(a,b) \in \tau_{\text{euclidea}}$ allora prendo 
	\begin{equation*}
	(a,b) = \bigcup_{n \in \N} \left[a + \frac{1}{n}, b\right)
	\end{equation*}
	Ma l'aperto della $\left[0,1\right) \in \tau_j$ non appartiene a $\tau_{\text{euclidea}}$ poiché non è della forma $(a,b)$. \\ Segue che $\tau_j$ è strettamente più fine di $\tau_{\text{euclidea}}$.
\end{proof}

Allora valgono alcune proprietà già viste nei precedenti capitoli, tra cui il fatto di non avere basi numerabili, avere sistemi di intorni numerabili, essere più fine della topologia euclidea standard, ed altro come vedremo in questa sezione.

\begin{theorem}
	Le seguenti affermazioni sono vere in $\tau_j$.
	\begin{enumerate}
		\item L'intervallo $\left[0,1\right]$ è un chiuso in $\tau_j$. 
		\item La funzione $f \colon x \mapsto -x$ non è continua in $\tau_j$.
	\end{enumerate}
\end{theorem}
\begin{proof}\
	\begin{enumerate}
		\item Per dimostrare la chiusura bisogna far vedere che $\left[0,1\right]^c$ sia aperto. Ma questo è abbastanza facile. Infatti $\left[0,1\right]^c = \left(-\infty, 0\right) \cup \left(1, +\infty\right)$, per cui basta vedere che $\left(-\infty, 0\right) = \bigcup_{n \in \N} \left[-n, 1/(n+1)\right)$ e $\left(1, +\infty\right) = \bigcup_{n \in \N} \left[1+1/(n+1), n+2\right)$, per cui risulta che $\left[0,1\right]$ è un chiuso.
		\item La funzione è ovviamente non continua, infatti si consideri l'aperto $\left[0,1\right)$, la sua controimmagine attraverso $f$ è $\left(-1, 0\right]$ che non è aperto.
	\end{enumerate}
\end{proof}

\begin{theorem}
	L'insieme $\mathcal{B} = \left\{ \left[a, b\right] \,\middle|\, a \le b, a, b \in \R \right\}$ è una base su $\R$ e la topologia generata è quella discreta. 
\end{theorem}
\begin{proof}
	Dimostro che $\mathcal{B}$ è un ricoprimento di $\R$. Infatti basta prendere $\R = \bigcup^{+\infty}_{n=1} \left[-n, n\right]$. Inoltre se interseco due chiusi nella topologia euclidea è ancora un chiuso quindi ho la stessa cosa qui. Per cui $\mathcal{B}$ è una base. \\ 
	
	È ovvio che tutti i singoletti di $\R$ sono contenuti in $\mathcal{B}$ quindi $\mathcal{C}\subset \mathcal{B}$ dove $\mathcal{C}$ genera la topologia discreta e poiché è la più fine dev'essere che anche $\mathcal{B}$ generi la topologia discreta su $\R$.
\end{proof}



\subsection{\textcolor{TopGener}{\textbf{Un breve esercizio}}}



\begin{example}
	Definito l'insieme $$\mathcal{B} = \left\{A \in 2^\R \,\middle|\, 0 \notin A \right\} \cup \left\{ A \in 2^\R \,\middle|\, 0 \in A, \R \setminus A \; \text{finito} \right\}$$ calcolare la chiusura di $J = \left[1,2\right] \times \left\{0\right\}$. \\ \\
	È ovvio dimostrare che l'insieme oltre ad essere una base è anche una topologia. \\ Inoltre vale $\overline{J} = \overline{\left\{0\right\}} \times \overline{\left[1,2\right]}$, quindi posso separare i casi in una topologia $\mathcal{B}$. Vediamo che $\overline{\left\{0\right\}} = \left\{0\right\}$ per la definizione della topologia, $\left[1,2\right]$ è aperto e $\left[1,2\right] \cup \left\{0\right\}$ è chiuso (perché il complementare è aperto). \\ Segue che $\overline{\left[1,2\right]} = \left[1,2\right] \cup \left\{0\right\}$. Quindi ho trovato $\overline{J}$.
\end{example}
\begin{proof}
	
\end{proof}



\section{Topologia di Zariski e gli insiemi algebrici}
\subsection{\textcolor{TopGener}{\textbf{Una topologia aliena in un mondo noto}}}
%todo: inserire spiegazione di roba noetheriena



\begin{definition}
	Si definisce topologia di Zariski su $\R^n$ la topologia generata dall'insieme dei chiusi 
	\begin{equation*}
	\mathcal{F} \coloneqq \left\{ C \in 2^{\R^n} \,\middle|\, C \; \text{insieme algebrico} \right\}
	\end{equation*}
\end{definition}

\begin{theorem}
	Gli insiemi algebrici formano una famiglia di chiusi che generano una topologia per $\R$.
\end{theorem}
\begin{proof}\
	\begin{enumerate}
		\item I polinomi $f(x) = 0$ e $f(x) = 1$ danno come soluzioni rispettivamente gli insiemi $\R, \varnothing$.
		\item Bisogna dimostrare che l'intersezione di insiemi algebrici è ancora un insieme algebrico. Poiché tutti gli insiemi algebrici in $\R$ sono tutti finiti\footnote{Supponi che $X \neq \R$ sia algebrico e infinito allora esiste un polinomio che ha $\deg(f) \ge |X|$ ma i polinomi non possono avere grado infinito, assurdo.} tranne $\R$, l'intersezione di insiemi finiti sarà un insieme finito, per cui posso enumerare le soluzioni $\left\{x_1, \dots, x_n\right\}$ e quindi generare il polinomio $f(z) = (z-x_1)\cdots(z-x_n)$. Se l'intersezione è infinita è algebrico perché è $\R$. 
		\item L'unione agisce in modo più prevedibile. Infatti se $F_1, \dots, F_n$ sono algebrici allora esistono $f_1, \dots, f_n$ che li generano. Per cui $f = f_1 \cdots f_n$ è il polinomio che genera $F_1 \cup \dots \cup F_n$.
	\end{enumerate}
\end{proof}

\begin{theorem}
	Gli insiemi algebrici formano una topologia su $\R^n$.
\end{theorem}
\begin{proof}
	Analogamente alla dimostrazione in $\R$ si dimostra che
	\begin{enumerate}
		\item I polinomi $f(x_1, \dots, x_n) = 0$ e $f(x_1, \dots, x_n) = 1$ danno come soluzioni rispettivamente gli insiemi $\R^n, \varnothing$.
		\item Siano $F_1, \dots, F_n$ insiemi algebrici allora esistono $f_1, \dots, f_n$ che li generano. Per cui $f = f_1 \cdots f_n$ è il polinomio che genera $F_1 \cup \dots \cup F_n$.
		\item Questa parte è meno banale. Userò la notazione tratta dal libro W. Fulton, Algebraic Curves. Dimostro inannzitutto che se $\left\{I_i\right\}_{i \in \mathcal{I}}$ dove $I_i$ è un ideale di $\R\left[x_1, \dots, x_n\right]$\footnote{questo anello è \textbf{Noetheriano} poiché $\R$ è un campo.} per ogni $i$, allora 
		\begin{equation*}
			\bigoplus_{i \in \mathcal{I}} I_i = I
		\end{equation*}
		dove $I$ è un ideale di $\R\left[x_1, \dots, x_n\right]$. \\ Poiché siamo in un anello Noetheriano $I$, questo è un ideale per qualsiasi insieme indicizzante $\mathcal{I}$. 
		Per cui 
		\begin{equation*}
		\bigcap_{i \in I} A_i = \bigcap_{i \in I} V(I_i) = V\left(\bigoplus_{i \in \mathcal{I}} I_i\right) = V(I)
		\end{equation*}
		e poiché $I$ è un ideale ho anche che $V(I)$ è un insieme algebrico per definizione. \footnote{dal punto di vista della notazione $A_i = V(I_i)$ sta ad indicare che l'insieme algebrico $A_i$ è l'insieme delle soluzioni dell'ideale $I_i$, visto che ogni insieme algebrico è generato da almeno un ideale di polinomi}
	\end{enumerate}	
\end{proof}



\section{Esempio di quoziente}
\subsection{\textcolor{TopGener}{\textbf{Un breve esempio di quoziente topologico}}
	


\begin{theorem}
	Si consideri $\left[0,1\right]$ con topologia euclidea indotta, dimostrare che $(\left[0,1\right]/R, \tau_{\text{euclidea}}|_{\left[0,1\right]/R}) \simeq (S^1, \tau_{\text{euclidea}}|_{S^1})$ dove $R$ è la relazione che associa $xRy \Leftrightarrow x = y \lor |x - y| = 1$
\end{theorem}
\begin{proof}
	Possiamo usare uno dei corollari (come il \textit{primo teorema di omeomorfismo}) costruiti negli spazi quozienti. Per cui ci basta trovare un identificazione $\morphism{f}{\left[0,1\right]}{\S^1}$ e la scelta più naturale ricade su 
	\begin{equation*}
		f(x) = (\cos 2\pi x, \sin 2 \pi x)
	\end{equation*}
	che è suriettiva, continua sulla topologia euclidea e vale $f^{-1}f(x) = \left[x\right]_R$. Meno banale è dimostrare che $\tau_f = \tau_{\text{euclidea}}|_{S^1}$. Usando il teorema \ref{thr:freeomeombyquo}
\end{proof}



\section{Discriminazione di topologie}
\subsection{\textcolor{TopGener}{\textbf{Discriminare, ma in senso buono}}}



Non è sempre facile dimostrare che due topologie sono la stessa a meno di omeomorfismi. \\ Per questo se si hanno degli strumenti per riuscire a discriminare quelle distinte \textit{al volo}, si può ridurre il tempo perso a dimostrare che due topologie distinte siano la medesima. \\ \\A tal fine nei capitoli precedenti si sono sviluppate alcune proprietà che sono invarianti per omeomorfismi e dunque degli invarianti topologici. Vediamo una semplice carrellata di come possono essere utilizzati per discriminare le topologie distinte

\begin{theorem}
	Le seguenti affermazioni sono vere:
	\begin{enumerate}
		\item $\left[0,1\right] \not\simeq (0,1)$
		\item $\left[0,1\right] \not\simeq \R$
		\item $\left(0,1\right] \not\simeq (0,1)$
		\item $\left[0,1\right]^2 \not\simeq \left[0,1\right]$
		% TODO $\R \simeq (0,1)$??? dal punto di vista della connessione e della compattezza sembrerebbe funzionare
	\end{enumerate}
\end{theorem}
\begin{proof}\
	\begin{enumerate}
		\item La compattezza è un'invariante per omeomorfismi, quindi se le due topologie fossero la medesima avrei che $(0,1)$ compatto se e solo se $\left[0,1\right]$ lo è. Ma $\left[0,1\right]$ è compatto, mentre $(0,1)$ non lo è. Pertanto non possono essere omeomorfi.
		\item Usando ancora una volta la compattezza come invariante topologico si vede che $\R$ non è compatto mentre $\left[0,1\right]$ lo è, pertanto non possono essere omeomorfi.
		\item La connessione è un altro invariante topologico. Infatti supponiamo che $\left(0,1\right]$ sia omeomorfo a $\left(0,1\right)$. Allora esiste $f(\left(0,1\right]) = \left(0,1\right)$ e $f(1) = x \in (0,1)$. Quindi se si considera $\morphism{f}{(0,1)}{(0,1)\setminus \left\{x\right\}}$ di sicuro $x$ è interno a $(0,1)$, quindi dev'essere che $(0,1) \setminus \left\{x\right\} = (0, x) \cup (x, 1)$. Bisogna far notare che $f$ è ancora un omeomorfismo tra $(0,1)$ e $(0,1)\setminus \left\{x\right\}$ (è ovviamente bigettiva e manda aperti in aperti anche se manca un punto). Eppure $(0,1)$ è connesso e quindi $f((0,1))$ dovrebbe essere connesso, ma non lo è. \\ Pertanto non può esistere $f$ omeomorfismo.
		\item Analogo al caso precedente. Infatti si prenda un punto interno di $x \in (0,1)$, indipendentemente da dove verrà mappato su $\left[0,1\right]^2$, $\left[0,1\right]^2 \setminus \left\{f(x)\right\}$ sarà ancora connesso\footnote{È ovvio che sarà ancora connesso: se $x$ viene mappato sul bordo, allora essendo $(0,1)^2$ connesso e $\left[0,1\right]^2$ connesso segue che qualsiasi $Z$, tale che $(0,1)^2 \subset Z \subset \left[0,1\right]^2$ è ancora connesso; se $x$ è interno invece posso sempre trovare una funzione continua che aggira $x$ e collega ogni paio di punti di $\left[0,1\right]^2 \setminus \left\{f(x)\right\}$}, mentre $\left[0,1\right] \setminus \left\{x\right\}$ non lo è. 
	\end{enumerate}
\end{proof}
