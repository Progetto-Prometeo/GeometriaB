\chapter{Calcolo differenziale}
Il calcolo differenziale è alla base dell'analisi in ogni campo, quindi è solamente giusto iniziare dalla nozione di \textit{funzioni differenziabili in senso complesso}.\\
Possiamo usare allora alcune nozioni note di analisi per lavorare nel campo complesso: è importante notare come non ci sia una biezione \enquote{perfetta} tra le proprietà delle funzioni nel piano reale e quelle nel piano complesso. Immergiamoci ora nel meraviglioso mondo dell'analisi complessa.
\newpage
\section{Differenziabilità in senso complesso}
\subsection{\textcolor{AnComp}{\textbf{Da $\mathbb{R}$ a $\mathbb{C}$: funzioni differenziabili}}}

Partiamo dal presupposto che possiamo identificare $\C$ con $\R^2$ dato che hanno la stessa topologia - e noi siamo interessati principalmente a questa struttura, dato che vogliamo definirne continuità e  differenziabilità in senso complesso. \\ Allora ogni funzione di una variabile complessa $\morphism{f}{\C}{\C}$ sarà essenzialmente trattata come se fosse una funzione $\morphism{f}{\R^2}{\R^2}$, ovvero ponendo $z = x + iy$
\begin{equation}
	f(z) = u(x,y) + iv(x,y)
\end{equation}
ma bisogna ricordare che le proprietà di $i$ ci daranno delle condizioni ulteriori sui criteri per cui possiamo definire $u, v$ e quindi ciò che diremo sui complessi non si potrà estendere direttamente sulle funzioni del tipo $\R^2 \to \R^2$. Possiamo comunque notare che $f$ è continua se e solo se $u,v$ lo sono (per proprietà universale del prodotto topologico).\\

Come nel caso reale in una variabile, possiamo esprimere la differenziabilità in funzione del limite del rapporto incrementale

\begin{definition}
	Sia $z \in \Omega \subset \C$. La funzione $f$ si dice \textbf{differenziabile} se esiste un numero complesso $f'(z)$ tale che per $h \to 0$ vale la seguente relazione
	\begin{equation}
		f(z+ h) - f(z) = f'(z)h + o(|h|)
	\end{equation} 
\end{definition}

\begin{definition}
	Sia $z \in \Omega \subset \C$. Se $f$ è differenziabile, la \textbf{derivata direzionale} di $f$ lungo $v \in \C$ (con $|v| = 1$) si definisce come
	\begin{equation}
		D_v f(z) = \lim_{r \to 0} \frac{f(z + rv) - f(z)}{r} = f'(z) v
	\end{equation}
\end{definition}


Inoltre, per come è stata definita la derivata, possiamo sfruttare tutte le proprietà che valgono nel caso reale di una variabile. Mostriamo alcuni esempi, come la derivata delle seguenti funzioni (se sono differenziabili \dots).

	\begin{equation*} 
	\begin{aligned}
		f(z) & = \log(|z|) 			& \text{non diff.} &  \\
		f(z) & = e^{iz - |z|} 		& \text{non diff.} & \\
		f(z) & = (1 + i - z)z^2  	& f'(z) & = 2(1+i)z - 3z^2 \\
	\end{aligned} 
	\end{equation*}
\\

Occupiamoci della derivabilità in senso complesso: la seguente definizione è fondamentale.

\begin{definition}
	Una funzione $\morphism{f}{\Omega \subset \C}{\C}$ è detta \textbf{olomorfa} se è differenziabile in ogni $z \in \Omega$. In generale indicheremo l'insieme delle funzioni olomorfe su $\Omega$ con $\mathcal{O}(\Omega)$.
\end{definition}

\begin{remark}
	Ogni funzione olomorfa è continua. Per dimostrarlo basta rispolverare la dimostrazione fatta nel caso dell'analisi reale.
\end{remark}

In virtù della relazione stabilità precedentemente, ovvero che $f = u + iv$ per delle funzioni $\morphism{u,v}{\R^2}{\R}$, otteniamo una condizione che ci eravamo limitati solo a ipotizzare.

\begin{theorem}[Equazioni di Cauchy-Riemann]
	Sia data una funzione complessa su $\Omega \subset \C$ del tipo $f(x+iy) = u(x,y) + iv(x,y)$ . Allora $f$ è differenziabile in $z \in \Omega$ se e solo se $u,v$ sono differenziabili e $u_x = v_y, u_y = - v_x$ (in $z \in \C$).
\end{theorem}
\begin{proof}
	Supponiamo $f$ differenziabile in $z=x+iy$. Allora consideriamo le derivate direzionali lungo l'asse reale e quello immaginario.
		\begin{equation*} \begin{aligned}
		D_{1} f  &= f'(z) = f_x(z) \\
		D_{i} f  &= f'(z)i = f_y(z)  
		\end{aligned}\end{equation*}
	dove $f_x(z) = u_x(x,y) + iv_x(x,y)$ e $f_y(z) = u_y(x,y) + iv_y(x,y)$. Da questo noto che vale la relazione
	\begin{equation*}
		f'(z) = u_x(z) + iv_x(x,y) = -i(u_y(x,y) + iv_y(x,y))
	\end{equation*}	
	ovvero $u_x(x,y) = v_y(x,y), v_x(x,y) = -u_y(x,y)$. Inoltre ci serve osservare che $u,v$ sono differenziabili in $z$: infatti poiché $f$ è differenziabile, allora anche la sua parte reale è differenziabile, come anche la sua parte immaginaria, e per $h = a + ib$ vale dunque 
	\begin{equation*}
	\begin{aligned}
		u(z+h) - u(z) & = u_x a - v_x b + o(\sqrt{a^2 + b^2})\\
		v(z+h) - v(z) & = u_x b + v_x a + o(\sqrt{a^2 + b^2})
	\end{aligned}
	\end{equation*}
	ovvero $u,v$ sono differenziabili in $z$, che era quanto si voleva dimostrare. 
	Supponiamo quindi che $u,v$ siano differenziabili in $z$ e valgano le relazioni sopra descritte. Allora con qualche calcolo si ottiene, sempre per $h = a + ib$
	\begin{equation*}
	\begin{aligned}
		u(z+h) - u(z) & = a u_x(z) + b u_y(z) + o(|h|) = a u_x(z) - b v_x(z) + o(|h|) \\
		v(z+h) - v(z) & = a v_x(z) + b v_y(z) + o(|h|) = a v_x(z) + b u_x(z) + o(|h|) \\
		\end{aligned}
	\end{equation*}
	da cui - ricordando la definizione di $f$ - si ottiene direttamente che questa è derivabile in senso complesso.
\end{proof}

\begin{corollary}
	La matrice delle funzioni reali $\morphism{(u,v)}{\Omega \subset \R^2}{\R^2}$ tali da descrivere una funzione complessa $f$, sono tali che  
	\begin{equation*}
	\begin{aligned}
		J_{(u,v)} = \left( \begin{array}{cc}
					u_x & - v_x\\ v_x & u_x
					\end{array} \right)
	\end{aligned}
	\end{equation*}
	In particolare il determinante è $\det J_{(u,v)} = u^2_x + v^2_x \ge 0$.
\end{corollary}

\begin{corollary}
	Se $f'(z) = 0$, $f$ è una funzione costante.
\end{corollary}

\begin{corollary}
	Se $f = u + iv$ con $u(x,y) = r \in \R$ e $f \in \mathcal{O}(\Omega)$, allora 
	\begin{equation*} 
	\begin{aligned}
		0 = u_x &= v_y \\
		0 = u_y &= -v_x
	\end{aligned}\end{equation*}
	E quindi risulta essere una funzione costante. Analogamente se $v(x,y) = p \in \R$ allora $f$ è una funzione costante.   
\end{corollary}

\subsection{\textcolor{AnComp}{\textbf{Funzioni armoniche ed operatori di Wirtinger}}}

\begin{definition}[Operatori di Wirtinger]
	Sia $\morphism{f}{\Omega}{\C}$ funzione di classe $C^1(\Omega)$. Allora possiamo definire i seguenti operatori
	\begin{equation*}
	\begin{aligned}
		\frac{\partial f}{\partial z} := \frac{1}{2}(f_x - if_y) \quad\ \frac{\partial f}{\partial \overline{z}} := \frac{1}{2}(f_x + if_y)
	\end{aligned}
	\end{equation*}
\end{definition}

\begin{remark}
	Si dimostra che una funzione soddisfa il teorema di Cauchy-Riemann sse vale 
	\begin{equation*}
		\frac{\partial f}{\partial \overline{z}} = 0
	\end{equation*}
	In particolare se $f$ olomorfa 
	\begin{equation*}
		\frac{\partial f(z)}{\partial z} = f'(z) 
	\end{equation*}
\end{remark}

\begin{remark}
	Se $f \in C^1(\Omega)$, allora soddisfa le equazioni di Cauchy-Riemann. In particolare se supponiamo $f = u+iv$ ulteriormente derivabile - cosa che non è necessario supporre, dato che le funzioni olomorfe, come vederemo in seguito, sono $C^\infty$ - possiamo costruire una relazione sulle funzione $u,v$:
	\begin{equation*}
	\begin{aligned}	
		u_{xx} & = (v_y)_x = (v_x)_y = (-u_y)_y = u_{yy} \\
		v_{xx} & = (-u_y)_x = -(u_x)_y = -(v_y)_y = -v_{yy}\\
	\end{aligned}
	\end{equation*}
	ovvero le due funzioni reali sono \textbf{armoniche}, dato che hanno operatore laplaciano nullo. In generale vale il seguente enunciato (che verrà dimostrato in seguito):\\
	
	\textit{Sia $\morphism{u}{\Omega \subset \R^2}{R}$ una funzione armonica su un dominio semplicemente connesso $\Omega$. Allora esiste una funzione armonica $\morphism{v}{\Omega}{\R}$ - unica a meno di una costante - tale che $f(z=x+iy) = u(x,y) + iv(x,y)$ sia una funzione olomorfa}.
\end{remark}		

\section{Funzioni analitiche o funzioni olomorfe?}
\subsection{\textcolor{AnComp}{\textbf{Serie e successioni in senso complesso}}}

Il termine \textit{analitico} si perde nel contesto delle funzioni complesse dato che, come vedremo, l'insieme delle funzioni olomorfe coincide con quello delle funzioni analitiche. \\ Pertanto si preferisce descrivere le funzioni analitiche come olomorfe - una condizione facilmente accertabile, data anche l'equivalenza stabilita delle equazioni di Cauchy-Riemann. \\ \\ In questa sezione ci limiteremo a sviluppare la teoria strettamente necessaria per stabilire la convergenza di serie complesse, arrivando infine a dimostrare quanto detto sfruttando il teorema di Abel.\\

\begin{definition}
	\label{defn:successione-di-funzioni-complesse}
	Sia $\Omega \subset \C$ e sia $\{f_n\}_{n\in\N}$ con $\morphism{f_n}{\Omega}{\C}$ funzione, allora $\{f_n\}_{n\in\N}$ si dice \textbf{successione di funzioni complesse}.
\end{definition}

\begin{definition}
	\label{defn:serie-di-funzioni-complesse}
	Data una successione di funzioni complesse $\{f_n\}_{n\in\N}$ definite su $\Omega\subset\C$, allora 
	\begin{equation*}
	\begin{aligned}
		\sum_{n=0}^{+\infty} f_n 
	\end{aligned}
	\end{equation*}
	si dice \textbf{serie di funzioni complesse}.
\end{definition}

\begin{definition}
	\label{defn:convergenza-puntuale-successione}
	Data una successione di funzioni complesse definite su $\Omega$, $\{f_n\}_{n\in\N}$, se per ogni $z \in \Omega$ vale 
	\begin{equation*}
	\begin{aligned}
		f_n(z) \to f(z) \quad\ \text{per}\ n \to +\infty
	\end{aligned}
	\end{equation*}
	per una qualche $\morphism{f}{\Omega}{\C}$, allora si dice che la successione \textbf{converge puntualmente} a $f$.
\end{definition}

\begin{definition}
	\label{defn:convergenza-puntuale-serie}
	Data una successione di funzioni complesse definite su $\Omega$, $\{f_n\}_{n\in\N}$, se per ogni $z \in \Omega$ vale 
	\begin{equation*}
	\begin{aligned}
		\sum^{n}_{k=0} f_k(z) \to f(z) \quad\ \text{per}\ n \to +\infty
	\end{aligned}
	\end{equation*}
	per una qualche $\morphism{f}{\Omega}{\C}$, allora si dice che la serie \textbf{converge puntualmente} a $f$.
\end{definition}

\begin{definition}
	\label{defn:convergenza-uniforme-successione}
	Data una successione di funzioni complesse definite su $\Omega$, $\{f_n\}_{n\in\N}$, se vale 
	\begin{equation*}
	\begin{aligned}
		\sup_{z \in \Omega} |f_n(z) - f(z)| \to 0 \quad\ \text{per}\ n \to +\infty
	\end{aligned}
	\end{equation*}
	per una qualche $\morphism{f}{\Omega}{\C}$, allora si dice che la successione \textbf{converge uniformemente} a $f$.
\end{definition}

\begin{definition}
	\label{defn:convergenza-uniforme-serie}
	Data una successione di funzioni complesse definite su $\Omega$, $\{f_n\}_{n\in\N}$, se vale 
	\begin{equation*}
	\begin{aligned}
		\sup_{z \in \Omega} \left|\sum^{N}_{n=0}f_n(z) - f(z)\right| \to 0 \quad\ \text{per}\ N \to +\infty
	\end{aligned}
	\end{equation*}
	per una qualche $\morphism{f}{\Omega}{\C}$, allora si dice che la serie \textbf{converge uniformemente} a $f$.
\end{definition}	
	
\begin{definition}
	\label{defn:convergenza-assoluta}
	Data una successione di funzioni complesse definite su $\Omega$, $\{f_n\}_{n\in\N}$, se per ogni $z \in \Omega$ vale 
	\begin{equation*}
	\begin{aligned}
		\sum^{+\infty}_{n=0} |f_n|(z) < +\infty \quad\ \text{per}\ n \to +\infty
	\end{aligned}
	\end{equation*}
	allora si dice che la serie \textbf{converge assolutamente}.	
\end{definition}

\begin{remark}
	Osserviamo infine che se una serie $\{\sum^n_{k=0}f_k\}_{n \in \N}$ è assolutamente convergente, allora è anche puntualmente convergente, infatti
	\begin{equation*}
		\left|\sum^n_{k=0} f_k(z)\right| \le \sum^n_{k=0} |f_k(z)| = l < +\infty
	\end{equation*} 
	per cui $\sum^n_{k=0} \Re(f_k(z)) < +\infty,\ \sum^n_{k=0} \Im(f_k(z)) < +\infty$. Per cui tutta la serie converge a qualche funzione $f$ in ogni punto.
\end{remark}

\begin{theorem}[M-test di Weierstrass]
	\label{thr:m-test-weierstrass} Sia $\{f_n\}_{n\in\N}$ una successione di funzioni complesse e $\{M_n\}_{n\in\N}$ una successione di numeri reali positivi per cui vale $|f_n(z)| \le M_n \forall z\in \Omega$. Se la serie 
	\begin{equation*}
		\sum_{n=0}^{+\infty} M_n < +\infty
	\end{equation*} 
	allora anche la serie 
	\begin{equation*}
		\sum_{n=0}^{+\infty} f_n(z)
	\end{equation*}
	converge assolutamente ed uniformemente in $\Omega$.
\end{theorem}
\begin{proof}
	Ovviamente che la serie converge in modo assoluto su tutto $\Omega$ per ipotesi. Allora cui definiamo $s_n(z) := \sum_{k=0}^n f_k(z)$ e $s$ la funzione a cui converge la serie. Allora
	\begin{equation*}
		|s - s_n| = \left|\sum_{k=n+1}^{+\infty} f_k\right| \le \sum_{k=n+1}^{+\infty} \left|f_k\right| \le \sum_{k=n+1}^{+\infty} M_k
	\end{equation*}
	Dato che la serie degli $M_k$ converge, fissato un qualsiasi $\varepsilon > 0$ possiamo sempre trovare $N$ tale per cui $\sum_{k>N} M_k < \varepsilon$. Per l'assoluta generalità con cui è stato scelto $z \in \Omega$ ottengo la tesi di uniforme convergenza. 
\end{proof}


\subsection{\textcolor{AnComp}{\textbf{Le serie di potenze}}}


\begin{definition}
	\label{defn:serie-di-potenze}
	Siano dati $z_0 \in \C$ e una successione $\{a_n\}_{n\in \N}$ di valori in $\C$, allora si definisce \textbf{serie di potenze} la serie della seguente forma
	\begin{equation*}
		S(z) := \sum_{n=0}^{+\infty} a_n(z-z_0)^n
	\end{equation*}
\end{definition}

\begin{theorem}[Criterio di Hadamard]
	\label{thr:criterio-hadamard}
	Data una serie di potenze $S$ con notazione come in Definizione \ref{defn:serie-di-potenze}, definiamo il suo \textbf{raggio di convergenza} come $R := ( \limsup_{n\to+\infty} \sqrt[n]{a_n})^{-1} \in \left[0,+\infty\right]$. Allora 
	\begin{enumerate}
		\item La serie di potenze $S$ converge assolutamente nel disco $B_{z_0}(R)$ e non converge per $z \notin \overline{B_{z_0}(R)}$.
		\item La serie di potenze converge su ogni disco chiuso $\overline{B_{z_0}(r)}$ con $r < R$.  
	\end{enumerate}
\end{theorem}
\begin{proof}
	Possiamo supporre $z_0 = 0$ - senza perdita di generalità, dato che basta effettuare una sostituzione per riportarci nel caso $z_0 = 0$ dato un qualsiasi $z_0 \in \C$.\\ 
	Dimostro che la serie di potenze converge uniformemente e assolutamente per ogni $0 < r < R$. Essendo $R < t^{-1}$ allora esiste un $N$ per cui
	
		\begin{equation*}
		\sqrt[n]{a_n} < t^{-1} \quad\ \text{per ogni}\ n \ge N 
		\end{equation*}
	
	dunque $|a_n| < t^{-n}$ per ogni $n \ge N$. \\ Se $|z| \le r$ vale
	
	\begin{equation*}
	|a_nz^n| < \left(\frac{r}{t}\right)^n
	\end{equation*}
	
	Per il Teorema \ref{thr:m-test-weierstrass} la serie di potenze converge $\forall z \in B_0(r)$ e siccome vale per ogni $r < R$ segue anche per $B_0(R)$.\\
	
	Dimostriamo ora che non converge da nessuna parte al di fuori della chiusura del disco aperto $\overline{B_{z_0}(R)}$. Infatti se supponessimo la convergenza in qualche punto $z \in \C \setminus \overline{B_{z_0}(R)}$ allora $|z| > R$ e $|z|^{-1} < R$. Per cui $\forall n \ge N$ con $N \in \N$ vale che
	
		\begin{equation*}
		|z|^{-1} < \sqrt[n]{|a_n|}
		\end{equation*}
		
	per cui otteniamo 
	\begin{equation*}	
	\sum^{+\infty}_{n=0}|z|^{-n}|z|^n = 	
		\sum^{+\infty}_{n=0}|\sqrt[n]{|a_n|}|^n|z|^n < 
		\sum^{+\infty}_{n=0}|a_n||z^n| 
	\end{equation*}
	
	e la prima serie è ovviamente non assolutamente convergente per ogni $|z| > R$.
\end{proof}

\begin{theorem}
	\label{prop:invarianza-raggio-convergenza-derivata}
	Data una serie di potenze definiamo la sua derivata come 
	\begin{equation}
	\begin{aligned}
		S'(z) = \sum^{+\infty}_{k=1} ka_k(z-z_0)^{k-1}
	\end{aligned}
	\end{equation} 
	e dunque i raggi di convergenza di $S'$ ed $S$ sono gli stessi.
\end{theorem}
\begin{proof}
	È un ovvia applicazione del calcolo dei limiti; definiamo con $b_n = (n+1)a_{n+1}$ i coefficienti della serie di potenze derivata. 
	\begin{equation*}
	\begin{aligned}
		\limsup_{n\to+\infty} \sqrt[n]{b_n} & = \limsup_{n\to+\infty} \sqrt[n]{(n+1)a_{n+1}} \\
											& = \limsup_{n\to+\infty} \sqrt[n]{(n+1)} \sqrt[n]{a_{n+1}} \\
											& = \limsup_{n\to+\infty} \sqrt[n]{a_{n+1}} = \limsup_{n\to+\infty} \sqrt[n]{a_{n}} 
	\end{aligned}
	\end{equation*}
	che è appunto lo stesso raggio di convergenza.
\end{proof}

\begin{theorem}[Teorema di Abel]
	\label{thr:teorema-di-abel}
	Sia $S$ una serie di potenze con raggio di convergenza $R > 0$. Allora $S(z)$ è una funzione olomorfa in $|z-z_0| < R$ e $S'$ è la sua derivata.
\end{theorem}
\begin{proof}
	Assumiamo ancora una volta che $z_0$ della serie di potenze sia $z_0 = 0$. Prendiamo quindi $z \in B_0(R)$ e sia $\delta > 0$ tale che $\overline{B_0(\delta)} \subset B_0(R)$. Prendiamo quindi un $h \in B_0(\delta)$. Allora
	\begin{equation*}
	\begin{aligned}
		\frac{f(z+h) - f(z)}{h} & = \sum^{+\infty}_{n=0} a_n \frac{(z+h)^n - z^n}{h} = \sum^{+\infty}_{n=0} a_n \frac{(z+h)^n-z^n}{z+h-z} \\
								& = \sum^{+\infty}_{n=0} a_n z^{n-1} \frac{\left(\frac{z+h}{z}\right)^n - 1}{\left(\frac{z+h}{z}\right)^n - 1} \\
								& = \sum^{+\infty}_{n=0} a_n \sum^{n-1}_{j=0} z^{n-1-j}(z+h)^j 
	\end{aligned}
	\end{equation*}
	Vogliamo mostrare che la serie sopra descritta converge uniformemente.
	\begin{equation*}
		|z+h| \le |z| + |h| \le |z| + \delta
	\end{equation*}
	Quindi possiamo scrivere 
	\begin{equation*}
		\left|a_n \sum^{n-1}_{j=0} z^{n-1-j}(z+h)^j \right| \le n|a_n|(|z|+\delta)^{n-1} 
	\end{equation*}
	Utilizzando il Teorema \ref{thr:m-test-weierstrass} con $M_n = n|a+n|(|z|+\delta)^{n-1}$ sappiamo che converge per ogni $|z|+\delta < R$, dato che dal Teorema \ref{prop:invarianza-raggio-convergenza-derivata} vale che la derivata di una funzione ha lo stesso raggio di convergenza della funzione stessa. Per cui converge uniformemente in ogni $|z| < R$. 
	Possiamo quindi calcolare il limite del rapporto incrementale, da cui si ottiene 
	\begin{equation*}
		\lim_{h\to 0} \frac{f(z+h) - f(z)}{h} = \sum^{+\infty}_{n=1} na_nz^{n-1}
	\end{equation*}
\end{proof}

