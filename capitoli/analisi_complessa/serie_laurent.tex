\chapter{Serie di Laurent}

\begin{definition}
  Una \emph{serie di Laurent} è una serie di funzioni della forma 
  \begin{equation*}
    \sum_{n=-\infty}^{+\infty} a_n(z-z_0)^n = \sum_{n=1}^{+\infty} a_{-m}
            \frac{1}{(z-z_0)^m} + \sum_{n=0}^{+\infty} a_n(z - z_0)^{n}
  \end{equation*}
con $a_n \in \C$ e per ogni $n \in \Z$ e $z_0 \in \C$ fissato e $z\in \C
\setminus \left\{ z_0 \right\}$.
  \label{def:serie_laurent}
\end{definition}

\begin{proposition}
  Esistono $r, R \in \left[ 0, +\infty \right]$ tali che la serie 
  \begin{equation*}
    \sum_{n=-\infty}^{+\infty} a_n(z-z_0)^n
  \end{equation*}
  converge assolutamente su una corona circolare 
  \begin{equation*}
    D \coloneqq \left\{ z \in \C \,\middle|\, r < |z-z_0| < R \right\}
  \end{equation*}
  e converge uniformemente su ogni corona chiusa
  \begin{equation*}
    K \coloneqq \left\{ z \in C \,\middle|\, r' \le |z-z_0| \le R' \right\}
  \end{equation*}
  con $r' > r$ e $R' < R$.
  \label{prp:convergenza_ass_unif_laurent}
\end{proposition}
\begin{proof}
  Osserviamo che la serie di Laurent si può scrivere come sommatoria di due serie di
  potenze
  \begin{align}
    \label{eq:serie_laurent_pos}
    \sum_{n=0}^{+\infty} a_n (z-z_0)^{n} \\
    \label{eq:serie_laurent_neg}
    \sum_{m=-\infty}^{0} a_{m} \frac{1}{(z-z_0)^{m}}
  \end{align}
  la seconda basta fare una sostituzione $w = 1/(z-z_0)$ e diventa una serie di
  potenze.
  Per il Teorema di Hadamard \ref{thr:criterio-hadamard} le serie di potenze
  hanno un raggio di convergenza, quindi denotiamo con $R, R_1 \in \left[ 0, 
  +\infty \right]$ rispettivamente il raggio di convergenza della serie di
  potenze \ref{eq:serie_laurent_pos} e della serie di potenze
  \ref{eq:serie_laurent_neg}. Ponendo $r = 1/R_1$ si ottiene il raggio di
  convergenza della parte della serie di potenze negativa (ovvero formula
  \ref{eq:serie_laurent_neg}) in funzione di $z$. Pertanto la serie di Laurent
  converge (la parte a indici negativi) assolutamente in 
  $\left\{z \in \C \,\middle|\, |z-z_0| > r \right\}$
  e uniformemente in $\left\{ z \in \C \,\middle|\, |z-z_0| \ge r' \right\}$ per
  ogni $r' > r$. Analogamente la parte a indici positivi converge assolutamente
  in $\left\{ z \in \C \,\middle|\, |z -z_0| < R  \right\}$ e uniformemente per
  ogni $R' < R$ in $\left\{ z \in \C \,\middle|\, |z -z_0| \le R'  \right\}$.
  Intersecando gli insiemi di convergenza per la serie a indici positivi
  e quella a indici negativi, otteniamo la tesi.
\end{proof}

\begin{remark}
  Se $r < R$ la corona $C_{z_0}(r,R) \coloneqq \left\{ z \in \C \,\middle|\,
  r < |z - z_0|  < R \right\}$ è non vuota e la serie di Laurent ha somma $f \in
  \mathcal{O}(C_{z_0}(r,R))$. Viceversa se $f$ è olomorfa su una corona si
  ottiene lo sviluppo di Laurent della funzione.
  \label{rmk:reverso_card_holo_func}
\end{remark}

\begin{theorem}
  Sia $f \in \mathcal{O}(\Omega)$ e $\Omega \subset \C$ aperto contenente la
  corona circolare chiusa 
  \begin{equation*}
    \overline{C_{z_0}(r,R)} \coloneqq \left\{ z \in \C \,\middle|\,
            \le < |z - z_0| \le R \right\}
  \end{equation*}
  allora si può scrivere $f$ come somma di una serie di Laurent, ovvero 
  \begin{equation*}
    f(z) = \sum_{n = -\infty}^{+\infty} a_n(z-z_0)^n
  \end{equation*}
  dove $z \in C_{z_0}(r,R)$. Vale la convergenza assoluta su $C_{z_0}(r,R)$ e la
  convergenza uniforme su $\overline{C_{z_0}(r',R')}$ con $r < r' < R' < R$.
  I coefficienti $a_n$ sono della seguente forma 
  \begin{equation*}
    a_n = \frac{1}{2\pi i} \int_{\gamma_{r''}} \frac{f(w)}{(w-z_0)^{n+1}}\ dw
  \end{equation*}
  dove $\gamma_{r''} = \partial B_{z_0}(r'')$ percorso in senso antiorario con
  $r \le r'' \le R$.
  \label{thr:function_as_laurent_series}
\end{theorem}
\begin{proof}
  Siano $R,r$ i raggi di convergenza della corona della serie di Laurent. Allora
  definiamo $\Gamma = \gamma_R - \gamma_r \sim_\Omega 0$. Sappiamo per omologia
  quindi che $z \in C_{z_0}(r,R)$ allora $1 = \Ind_{\Gamma}(z) = 
  \Ind_{\gamma_R}(z) - \Ind_{\gamma_r}(z)$ e per definizione dell'indice 
  sappiamo che $\Ind_{\gamma_R} = 1$ e quindi $\Ind_{\gamma_r} = 0$. 
  Per cui per il Teorema delle Formule Integrale di Cauchy
  \ref{thr:formule-di-cauchy-generale} si ottiene che 
  \begin{equation*}
    f(z) = f(z)\Ind_{\Gamma}(z) = \frac{1}{2\pi i} \int_{\gamma_R}
    \frac{f(w)}{w-z}\ dw - \frac{1}{2\pi i}\int_{\gamma_r} \frac{f(w)}{w-z}\ dw
  \end{equation*}
  e il primo integrale, in modo analogo a quanto fatto nel Teorema di
  Weierstrass \ref{thr:weierstrass-analitic-local-form} diventano i termini positivi 
  della serie di Laurent, ovvero
  \begin{equation*}
    \frac{1}{2\pi i }\int_{\gamma_R} \frac{f(w)}{w-z}\ dw
    = \sum_{n=0}^{+\infty} a_n (z-z_0)^{n}
  \end{equation*}
  Il secondo integrale si sviluppa in modo analogo solo che essendoci il meno
  cambia il segno
  \begin{equation*}
    - \int_{\gamma_r} \frac{f(w)}{w-z}\ dw = \int_{\gamma_r}
    \frac{f(w)}{z-z_0} \frac{1}{1- \frac{w-z_0}{z -z_0}}\ dw
  \end{equation*}
  possiamo vedere il secondo membro della moltiplicazione come una serie
  geometrica e quindi si ottiene che 
  \begin{equation*}
    -\int_{\gamma_r} \frac{f(w)}{w-z_0}\ dw = \sum_{n=-\infty}^{0}
    \int_{\gamma_r} \frac{f(w)}{(w-z_0)^{n+1}}\ dw (z-z_0)^n
  \end{equation*}
  Per concludere basta osservare che per ogni $\partial B_{z_0}(r')$ con $r \le
  r' \le R$ si ha che $\gamma_r \sim_{\Omega} \gamma_{r'} \sim_{\Omega}
  \gamma_R$ e quindi si ottiene che ogni $a_n$ si può scrivere come 
  \begin{equation*}
    a_n = \frac{1}{2\pi i} \int_{\gamma_{r'}} \frac{f(w)}{(w-z_0)^{n+1}}\ dw
  \end{equation*}
\end{proof}

\begin{remark}
  Al posto di $\gamma_{r'}$ nella forma di $a_n$ può essere presa una qualsiasi
  curva omotopa al disco. In partiolare basta che sia una curva di Jordan.
  \label{rmk:cambiamento_bordo_di_integrazione}
\end{remark}

\begin{remark}
  Se $f$ è olomorfa sul disco $B_{z_0}(R)$ i coefficienti $a_n$ con $n < 0$ sono
  tutti nulli, infatti la funzione $f(w) / (w-z_0)^{n+1}$ è olomorfa sul disco
  e quindi ammette primitiva ed essendo che il disco è una curva chiusa si ha
  che tutti i coefficienti a indice negativo della serie di Laurent si
  annullino.
  \label{rmk:coefficienti_negativi}
\end{remark}

\begin{definition}
  Il coefficiente della serie di Laurent $a_{-1}$ è detto \emph{residuo} di $f$
  in $z_0$ 
  \begin{equation*}
    a_{-1} \coloneqq \frac{1}{2\pi i} \int_{\gamma} f(w)\ dw
  \end{equation*}
  per ogni curva di Jordan $\gamma$ in $\Omega$ e $z_0 \in
  \operatorname{interno}(\gamma)$.
  \label{def:residuo}
\end{definition}

\section{Singolarità}

\begin{definition}
  Sia $D$ un disco aperto centrato in $z_0$. Se $f \in \mathcal{O}(D \setminus
  \left\{ z_0 \right\})$ si dice che $z_0$ è una \emph{singolarità isolata} di
  $f$.
  \label{def:singolarità_isolata}
\end{definition}

\begin{example}
  \begin{enumerate}
    \item Se una singolarità è eliminabile allora è anche isolata, dato che $f$
      è olomorfa ovunque tranne nel punto dove ha una singolarità.
  \end{enumerate}
\end{example}

\begin{definition}[Classificazione singolarità isolate]
  Sia $z_0$ singolarità isolata di $f \in \mathcal{O}(\Omega \setminus \left\{
  z_0 \right\})$ e siano $\{a_n\}_{n=-\infty}^{+\infty}$ i coefficienti della
  rispettiva serie di Laurent. Allora
  \begin{enumerate}
    \item $z_0$ è \emph{eliminabile} sse $a_n = 0$ per ogni $n < 0$.
    \item $z_0$ è un \emph{polo di ordine} $m > 0$ di $f$ sse $a_{-m} \neq 0$ e $a_n
      = 0$ per ogni $n < -m$.
    \item $z_0$ è una \emph{singolarità essenziale} sse esistono infiniti coefficienti
      $a_n \neq 0$ con $n < 0$.
  \end{enumerate}
  \label{def:classificazione_singolarita_per_serie_di_laurent}
\end{definition}

\begin{remark}
  Sia $f \in \mathcal{O}(\Omega \setminus \left\{ z_0 \right\})$ tale che ha serie di 
  Laurent $a_n = 0$ per ogni $n < 0$  allora $f$ può essere estesa come una 
  funzione olomorfa su tutto $\Omega$ (teorema di distruzione della singolarità).
  Quindi il suo sviluppo di Laurent equivale a quello della sua estensione.\\
  
  Se $f$ invece ha una singolarità eliminabile, allora posso \emph{eliminare} la
  singolarità e definire $\tilde{f}$ tale che è uguale $f$ ovunque e olomorfa su
  tutto $\Omega$. Quindi 
  \begin{equation*}
    a_n = \frac{1}{2\pi i} \int_{\gamma_R} \frac{f(w)}{w-z_0}\ dw 
        = \frac{1}{2\pi i} \int_{\gamma_R} \frac{\tilde{f}(w)}{w-z_0}\ dw
  \end{equation*}
  ma poiché $\tilde{f} \in \mathcal{O}(\Omega)$ segue che $a_n = 0$ per il teorema
  della forma integrale di Cauchy.
  \label{rmk:equivalenza_definizioni_sing_eliminabili}
\end{remark}

\begin{example}
  \begin{enumerate}
    \item Sia $f(z) = 1/z$ allora proviamo a vedere che tipo di singolarità ha
      in $z = 0$, il residuo è 
      \begin{equation*}
        % TODO: check if my calcs are correct
        \operatorname{Res}_f(0) = \frac{1}{2\pi i } \int_{\gamma} f(w)\ dw = 1
      \end{equation*}
      Per per ogni $n < 1$ i coefficienti si azzerano. Quindi ha un polo di ordine
      $1$.
    \item Sia invece 
      \begin{equation*}
        f(z) = \frac{z}{(\cos(z) - 1)^2}
      \end{equation*}
      allora stimando $\cos(z) - 1 = -z^2/2 + o(z^3) = z^2 h(z)$ con $h(0)
      = -1/2$ e $h \in \mathcal{O}(\Omega)$. Allora 
      \begin{equation*}
        f(z) = \frac{z}{z^4h(z)^2} = \frac{1}{z^3} \frac{1}{h(z)^2}
      \end{equation*} 
      con $1/h^2$ olomorfa su un intorno di $z = 0$. Quindi possiamo osservare
      che $1/h^2 = \sum_{n=0}^{+\infty} b_n z^n$ per qualche $b_n \in \C$.
      Infine si ottiene che 
      \begin{equation*}
        f(z) = \frac{b_0}{z^3} + \frac{b_1}{z^2} + \cdots
      \end{equation*}
      ovvero $f$ ha un polo di ordine $3$ nell'origine.
  \end{enumerate}
\end{example}

\begin{proposition}
  Una funzione $f \in \mathcal{O}(D \setminus \{z_0\})$ definita su un disco
  aperto $D$ ha un polo di ordine $m$ in $z_0$ se e solo se $g \coloneqq 1/f$ 
  è olomorfa in un intorno di $z_0$ e ha \emph{uno zero di ordine $m$} in $z_0$.
  \label{prp:caratterizzazione_poli}
\end{proposition}
\begin{proof}
  Dimostriamo che se $f$ ha un polo di ordine $m$ in $z_0$ allora $g$ è olomorfa in un
  intorno di $z_0$ e ha uno zero di ordine $m$. Essendo olomorfa allora per la
  decomposizione in serie di Laurent possiamo osservare che $h(z) \coloneqq
  f(z)(z-z_0)^m$ (è olomorfa su tutto $D$ questa), allora ha sviluppo solo in serie 
  di potenze. In particolare   
  \begin{equation*}
    h(z_0) = \lim_{z \to z_0} f(z)(z-z_0)^{m} = a_{-m}  
  \end{equation*}
  Quindi $g(z) = 1/f(z) = (z-z_0)^m 1/h(z)$ è olomorfa in un intorno $I$ di $z_0$.
  E ovviamente ha sviluppo in serie di potenze in $I$, ovvero ha forma
  \begin{equation*}
    g(z) = \frac{1}{a_{-m}} (z-z_0)^{m} + \cdots
  \end{equation*}
  ovvero ha uno zero di oridne $m$ in $z_0$.\\

  Dimostriamo l'implicazione nel senso contrario, anche se è del tutto analogo
  a quanto fatto precedentemente. Allora è ovvio che 
  \begin{equation*}
    g(z) = (z-z_0)^m \tilde{h}(z)
  \end{equation*}
  con $\tilde{h}(z_0) \neq 0$ (altrimenti avrei uno zero di ordine maggiore di
  $m$). Per definizione di $g$ si può scrivere 
  \begin{equation*}
    f(z) = \frac{1}{g(z)} = (z-z_0)^{-m} \frac{1}{\tilde{h}(z)}
  \end{equation*}
  poiché $g$ era olomorfa in un intorno di $z_0$, dev'essere che $\tilde{h}$
  abbia come sviluppo di Laurent uno sviluppo in serie di potenze, per cui 
  \begin{equation*}
    f(z) = a_{-m} (z-z_0)^{-m} + \cdots
  \end{equation*}
  e poiché $a_{-m} \neq 0$ segue che $f$ ha un polo di ordine $m$ in $z_0$.
\end{proof}

\begin{remark}
  Se $z_0$ è un polo di ordine $m > 0$ di $f$ e $f(z)(z-z_0)^m = h(z)$, allora
  $h \in \mathcal{O}(D)$ poiché la sua espansione di Laurent è una serie di
  potenze  
  \begin{equation*}
    \lim_{z\to z_0} \left|\frac{h(z)}{(z-z_0)^m}\right| = \lim_{z\to z_0} |f(z)|
    = + \infty
  \end{equation*}
  quindi segue che $h(z) = f(z)(z-z_0)^m = c \in \C$ finito per $z \to z_0$. 
  \label{rmk:limitatezza_tolto_il_polo}
\end{remark}

\begin{definition}
  Sia $S \subset \Omega$ sottoinsieme discreto dell'aperto $\Omega$. Se $f \in
  \mathcal{O}(\Omega \setminus S)$ e $f$ ha poli o singolarità eliminabili nei
  punti di $S$, allora $f$ è detta \emph{meromorfa} su $\Omega$. Lo indicheremo
  con $f \in \mathcal{M}(\Omega)$.
  \label{def:meromorfismo}
\end{definition}

\begin{example}
  \begin{enumerate}
    \item Se funzione razionale $f= p / q$ dove $p,q$ sono polinomi complessi
      e ovviamente $q \neq 0$ allora $f \in \mathcal{M}(\Omega)$. Infatti se
      $z_0$ è uno zero di $q$ di ordine $k$ allora possiamo supporre che al
      massimo per $l \ge 0$ vale 
      % TODO: formattare meglio questa espressione
      \begin{align*}
        & p(z) = (z-z_0)^l \tilde{p}(z) &  & q(z)  = (z-z_0)^k \tilde{q}(z)  \\
        & & \Longrightarrow f(z)  = (z-z_0)^{l-k}
        \frac{\tilde{p}(z)}{\tilde{q}(z)} 
      \end{align*}
      e quindi al massimo $z_0$ è un polo di ordine $l-k$ (se $k > l$)
      altrimenti è una singolarità eliminabile. 
    \item Sia $f(z) = e^{1/z}$ allora possiamo vedere che ha una singolarità
      essenziale all'origine
      \begin{equation*}
        e^{1/z} = \sum_{0}^{+\infty} \frac{1}{n!}\frac{1}{z^n} = \sum_{m
        = -\infty}^{0} \frac{1}{(-m)!} z^m 
      \end{equation*}
  \end{enumerate}
\end{example}

\begin{theorem}
    Se $f \in \mathcal{O}(D \setminus \left\{z_0  \right\})$ e $z_0$ è una
    singolarità essenziale di $f$ allora $f(D \setminus \left\{ z_0 \right\})$
    è denso in $\C$.
  \label{thr:casorati_weierstrass}
\end{theorem}
\begin{proof}
  Supponiamo che $\overline{f(D \setminus \left\{ z_0 \right\})} \neq \C$ allora
  esiste $\alpha \in \C$ e $\delta > 0$ tale che $B_{\alpha}(\delta) \cap f(D
  \setminus \left\{ z_0 \right\}) = \varnothing$ cioè $|f(z) - \alpha| \ge
  \delta$ per ogni $z \in D \setminus \left\{ z_0 \right\}$. Sia quindi
  \begin{equation*}
    g(z) = \frac{1}{f(z) - \alpha} \in \mathcal{O}(D \setminus \left\{ z_0
      \right\}
  \end{equation*}
  allora $g \le 1/\delta$, ovvero è limitata. Pertanto $g$ ha una singolarità
  eliminabile in $z_0$ e questo indica che $f - \alpha$ può avere al massimo una
  discontinuità eliminabile o un polo. Ma $f$ ha una singolarità essenziale, da
  cui l'assurdo.
\end{proof}


\begin{remark}
  Le singolarità isolate possono essere quindi caratterizzate dal
  \emph{comportamento locale} di $f$ in $z_0$: 
  \begin{enumerate}
    \item $z_0$ è una singolarità eliminabile sse esiste il limite e 
      $\lim_{z \to z_0} |f(z)| < +\infty$.
    \item $z_0$ è polo di $f$ sse esiste i limite e vale $\lim_{z\to
      z_0}|f(z)| = +\infty$
    \item $z_0$ è singolarità essenziale sse non esiste il limite
      $\lim_{z \to z_0} f(z)$. 
  \end{enumerate}
  \label{rmk:comportamento_locale_singolarita}
\end{remark}
