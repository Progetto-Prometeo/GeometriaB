
\chapter{Estensione di funzioni in $\C$}

\section{Funzione esponenziale}

\begin{definition}
	\label{defn:funzione-esponenziale}
	Definiamo l'esponenziale in $\C$ come la serie di potenze
	\begin{equation}
	\begin{aligned}
		e^z := \sum^{+\infty}_{k=0} \frac{z^k}{k!}
	\end{aligned}
	\end{equation}
\end{definition}

Possiamo osservarne alcune caratteristiche fondamentali. 

\begin{theorem}
	La funzione esponenziale gode delle seguenti proprietà:
	\begin{enumerate}
		\item è olomorfa in $\C$
		\item $(e^z)' = e^z$
		\item $e^{i\theta} = \cos \theta + i\sin \theta$ per $\theta \in \R$
		\item è periodica di periodo $2\pi i$
		\item $e^z \neq 0$ per ogni $z \in \C$ (in particolare non è biunivoca)
	\end{enumerate}
\end{theorem}
\begin{proof} \
	\begin{enumerate}
		\item Per definizione e per il Teorema \ref{thr:teorema-di-abel}.
		\item Dal calcolo esplicito della derivata della serie di potenze:
		$$\begin{aligned} (e^z)' & = \sum^{+\infty}_{n=1} n\frac{z^{n-1}}{n!} = \sum^{+\infty}_{n=1} \frac{z^{n-1}}{(n-1)!} \\
		& = \sum^{+\infty}_{p=0} \frac{z^p}{p!} = e^z \end{aligned}$$
		\item Anche per questo basta un calcolo esplicito: \begin{equation*}
		\begin{aligned}
		e^{iz} & = \sum^{+\infty}_{n=0} \frac{(iz)^{n}}{n!} = \sum^{+\infty}_{k=0} \frac{(iz)^{2k}}{2k!} + \sum^{+\infty}_{k=0} \frac{(iz)^{2k+1}}{2k+1!}\\
		& = \sum^{+\infty}_{k=0} (-1)^k\frac{z^{2k}}{2k!} + i\sum^{+\infty}_{k=0} (-1)^k\frac{z^{2k+1}}{2k+1!}\\
		& = \cos z + i \sin z
		\end{aligned}
		\end{equation*}
		\item Basta osservare che $e^{a + 2\pi i} = e^a e^{2\pi i } = e^a$, il fatto che $e^{2\pi i} = 1$ è naturale dal punto precedente.
		\item 	Sia $\alpha \in \C$, allora $e^z = \alpha$ vuol dire che se $z = x+iy$ allora 
		\begin{equation*}
		\begin{cases}
		x = \ln |\alpha|\\
		y = \arg \alpha
		\end{cases}
		\end{equation*}
		per cui se $\alpha = 0$, $x$ non sarebbe ben definito. Mentre per tutti gli altri valore di $\alpha \in \C$ è definita.
		
	\end{enumerate}
\end{proof}

\section{Funzioni trigonometriche}
	
	\begin{definition}
		\label{defn:sin-cos}
		Definiamo le funzioni trigonometriche in funzione delle serie di potenze come segue:
		\begin{equation}
		\begin{aligned}
			\sin(z) := \sum^{+\infty}_{k=0} (-1)^k\frac{z^{2k+1}}{(2k+1)!} \;\ \cos(z) := \sum^{+\infty}_{k=0} (-1)^k\frac{z^{2k}}{(2k)!}
		\end{aligned}
		\end{equation}
	\end{definition}
	
	\begin{theorem}
		Valgono le seguenti proprietà delle funzioni trigonometriche:
		\begin{enumerate}
			\item 
			\begin{equation*}
			\begin{aligned}
				\sin(z) = \frac{e^{iz} - e^{-iz}}{2} \;\ \cos(z) = \frac{e^{iz} + e^{-iz}}{2}
			\end{aligned}
			\end{equation*}
			\item sono $2\pi$ periodiche
			\item non sono limitate
			\item l'immagine di $\sin$ e $\cos$ coincide con $\C$
			\item le seguenti funzioni sono olomorfe tranne nei loro rispettivi poli
			\begin{equation*}
			\begin{aligned}
				\tan z & := \frac{\sin z}{\cos z} \in \mathcal{O}(\C \setminus \{ \pi /2 + k\pi \mid k \in \N \}) \\
				\cot z & := \frac{\cos z}{\sin z} \in \mathcal{O}(\C \setminus \{k\pi \mid k \in \N \})
			\end{aligned}
			\end{equation*}
		\end{enumerate}
	\end{theorem}
	\begin{proof} \ %TODO: proof 
	\begin{enumerate}
		\item 
		\item
		\item
		\item
		\item
	\end{enumerate}
	\end{proof}
	
\section{Funzione logaritmo}
	
	\begin{definition}
		\label{defn:logaritmo-principale}
		Il \textbf{logaritmo principale} è la funzione definita come:
		\begin{equation}
		\begin{aligned}	
			 & \morphism{\log}{\C \setminus \{x \in \R \mid x \le 0\}}{\C} \\
			 \log(z) = & \ln(|z|) + i\arg z \ \text{  con  } \ \arg z \in (-\pi,\pi]
		\end{aligned}
		\end{equation}
		
	\end{definition}

	\begin{remark}
		In generale si perdono molte delle proprietà del logaritmo che valevano in $\R$. Per esempio $\log(xy) \neq \log(x)+\log(y)$.
	\end{remark}

	\begin{theorem}
		La funzione logaritmo principale è olomorfa e la sua derivata è $$\log'(z) = 1/z$$.
	\end{theorem}
	\begin{proof} \
		Dimostriamo che è olomorfa grazie alle equazioni di Cauchy-Riemann, per cui ponendo
		\begin{equation*}
		\begin{aligned}
			\log(x+iy) = \frac{1}{2}\ln(x^2 + y^2) + i\arctan(y/x) = u(x,y) + iv(x,y)
		\end{aligned}
		\end{equation*}
		da cui le derivate ci danno le relazioni cercate.\\
		
		Sempre utilizzando le equazioni di Cauchy-Riemann otteniamo al tesi, cioè che $\log'(z) = u_x + iv_x = 1/z$.
	\end{proof}
	\begin{remark}
		Osserviamo infine che la funzione argomento può prendere valori con periodicità di $2\pi$, perciò abbiamo effettivamente infinite definizioni della funzione logaritmo. \\ Questo può presentare alcuni vantaggi. Infatti la funzione radice quadrata principalmente è 
		$f_0(z) = e^{\log(z)/2} = \sqrt{|z|}e^{i(\arg z)/2}$, ma se si usa un'altra definizione di logaritmo, ovvero consideriamo una diversa possibile branca (o determinazione) della radice quadrata, otteniamo $f_1(z) = e^{\log'(z)/2} = \sqrt{|z|}e^{i\arg(z)/2 + i\pi}$, che rappresenta la \textit{radice "negativa"}. 
	\end{remark}
