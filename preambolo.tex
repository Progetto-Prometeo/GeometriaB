% --- FONT & STYLE ------------------------------------
\usepackage[utf8]{inputenc}
\usepackage[italian]{babel}
\usepackage[T1]{fontenc}
\usepackage[dvipsnames]{xcolor} % devo caricarlo qui per option clash
\usepackage{nameref}

\usepackage{fancyhdr}
\definecolor{PageNum}{rgb}{0.09, 0.45, 0.27}
\fancypagestyle{plain}{%
	\fancyhf{}
	\fancyhead[LE,RO]{\slshape{    }}
	\fancyhead[RE,LO]{\textbf{Geometria B}}
	\fancyfoot[CE,CO]{\leftmark}
	\fancyfoot[LE,RO]{\textcolor{PageNum}{\textbf{\thepage}}}
}
\pagestyle{fancy}
\fancyhf{}
\fancyhead[LE,RO]{\slshape{    }}
\fancyhead[RE,LO]{\textbf{Geometria B}}
\fancyfoot[CE,CO]{\leftmark}
\fancyfoot[LE,RO]{\textcolor{PageNum}{\textbf{\thepage}}}

\renewcommand{\headrulewidth}{2pt}
\renewcommand{\footrulewidth}{1pt}
\renewcommand\footnoterule{}


%TODO: verificare che questo risolva i problemi di spaziatura e non dia fastidio 
%\newtheoremstyle{noice} % theoremstyle personalizzato
%{12pt}{18pt} % spazio sopra / sotto
%{}{} % font corpo ed indentatura
%{\bfseries}{:} % font teorema e roba dopo teorema
%{6 pt} % spazio dopo nome teorema
%{\thmname{#1}\thmnumber{ #2}\thmnote{ (#3)}} %roba inutile



% --- PACKAGES ------------------------------------
\usepackage{amsthm, amsmath, amssymb, mathtools}
\usepackage{tikz-cd}    % TODO: non so se servirà, per ora l'ho messo.
\usepackage{derivative} % Serve: il comando che era dv è diventato odv
\usepackage{float}

%packages per la copertina:
\usepackage{tikz}
% questo sotto da errore ma a compilation time gira!
\usetikzlibrary{shapes.geometric}
\usetikzlibrary{calc}
\usepackage{anyfontsize}
\usepackage[activate={true,nocompatibility},
final,
tracking=true,
kerning=true,
spacing=true,
factor=1100,
stretch=10,
shrink=10,
selected=true,
verbose=errors,
babel=true]{microtype} % apparenza testo
\usepackage[autostyle,
german=guillemets,
italian=quotes]{csquotes} % usa \enquote{} per le virgolette


% --- MISC ------------------------------------
\usepackage{hyperref}
\usepackage{booktabs}
\usepackage{dsfont}
% \usepackage{mnsymbol} % Note: se è possibile non usarlo



% --- DEFINITIONS (sets, operators) ------------------------------------
\def \Z {\mathbb{Z}}
\def \R {\mathbb{R}}
\def \C {\mathbb{C}}
\def \Q {\mathbb{Q}}
\def \N {\mathbb{N}}
\def \S {\mathbb{S}}
\def \D {\mathbb{D}}
\def \Ind {\operatorname{Ind}}
\def \image {\operatorname{Im}}



% --- FUNCTIONS, THEOREMS ------------------------------------
% Routine to write a general function given 
% function name, domain and codomain.

\newcommand{\morphism}[3]{
	#1 \colon\ #2 \rightarrow #3}

\newcommand{\diam}[1]{	% Function diameter of a set
	\operatorname{diam}\left(#1\right)}

\newcommand{\dist}[2]{	% Distance between "things"
	\operatorname{dist}\left(#1, #2\right)
}

% Theorems, definitions and styles 
%\theoremstyle{noice}
\theoremstyle{definition}
\newtheorem{theorem}{Teorema}[chapter]
\newtheorem{lemma}[theorem]{Lemma}
\newtheorem{proposition}{Proposizione}[chapter]
\newtheorem{corollary}[theorem]{Corollario}
\newtheorem{counterexample}[theorem]{Controesempio}

%\theoremstyle{noice}
\theoremstyle{definition}
\newtheorem{definition}[theorem]{Definizione}
\newtheorem{example}[theorem]{Esempio}
\newtheorem{xca}[theorem]{Esercizio}

%\theoremstyle{noice}
\theoremstyle{definition}
\newtheorem{remark}[theorem]{Osservazione}

\numberwithin{section}{chapter}
\numberwithin{equation}{chapter}

%    For a single index; for multiple indexes, see the manual
%    "Instructions for preparation of papers and monographs:
%    AMS-LaTeX" (instr-l.pdf in the AMS-LaTeX distribution).


