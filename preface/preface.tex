\chapter* {Introduzione: Un dessert prima di pranzo}

Queste note seguiranno gli appunti delle lezioni di mezzo semestre, e saranno principalmente dedicati all'analisi complessa per il corso di laurea triennale in Matematica dell'Università di Trento - in particolare il primo con l'introduzione dell'insegnamento a distanza per causa del Coronavirus. \\
Ovviamente la comprensione delle sezioni precedenti è prerequisito per questa, senza la quale risulterà difficile - ed a tratti persino \textit{complesso} - avere chiaro quanto stiamo per svolgere insieme. E dopotutto, la topologia algebrica è bellissima! \\ \\
Sicché si dice che la matematica è un enorme battuta che nessuno capisce e di conseguenza vada spiegata lentamente, 
seguiremo esattamente questa procedura nella stesura delle note: in principio introdurremo i teoremi in modo puramente formale, esprimeranno la propria bellezza ed eleganza in alcuni risultati 
e più avanti verranno propriamente motivati.

\section{Il meraviglioso mondo complesso}

Lo scopo di questa sezione è mostrare alcune proprietà che rendono il campo complesso $\C$ magnifico e addirittura \textit{semplice}.

\begin{definition}
	Una \textbf{funzione olomorfa} è una funzione $\morphism{f}{A \subset \C}{\C}$, dove $A$ è un aperto, differenziabile in senso complesso. 
\end{definition}

Ecco alcuni enunciati che ci semplificheranno la vita mentre lavoreremo
\begin{theorem}[Integrazione lungo curve]
	Se le curve $\alpha$, $\beta$ - con $\alpha \sim \beta$ - congiungono due punti di $A \subset \C$ allora 
	\begin{equation*}
	\int_\alpha f = \int_\beta f
	\end{equation*}
\end{theorem}

\begin{theorem}[Regolarità]
	Se $f$ è olomorfa allora $f \in C^{\infty}(A)$, dove $A\subseteq \C$ è il dominio di $f$.
\end{theorem}

\begin{theorem}[Principio di identità]
	Sia $A\subset \C$ tale che $A \simeq \D$ per omotopie, allora se $f(x) = g(x)$ per ogni $x\in A$, si ha che  $f = g$.
\end{theorem}
