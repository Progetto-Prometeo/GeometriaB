\chapter{Un dessert prima di pranzo}

Praticamente queste note seguiranno gli appunti delle lezioni di mezzo semestre dedicati all'analisi complessa del corso di laurea triennale di Trento a. s. il primo con il coronavirus e con l'introduzione dei mezzi digitali.
Sicché si dice che la matematica è un enorme battuta che nessuno capisce e poi va spiegata lentamente, 
queste note seguiranno lo stesso corso. Prima verranno introdotti alcuni teoremi che esprimeranno la bellezza di alcuni risultati 
e più avanti verranno motivati propriamente.

\section{Il meraviglioso complesso mondo}

Premessa di questa sezione è di mostrare alcune proprietà che rendono il campo complesso $\C$ magnifico e addirittura \textit{semplice}.

\begin{definition}
	Una \textbf{funzione olomorfa} è una funzione tale che $\morphism{f}{A \subset \C}{\C}$, dove $A$ aperto, differenziabile in senso complesso. 
\end{definition}

Ecco alcuni enunciati che semplificano la vita mentre si lavora con funzioni olomorfe in campo complesso.
\begin{theorem}[Integrazione lungo curve]
	Se $\alpha \sim \beta$ congiungono due punti di $A \subset \C$ allora 
	\begin{equation}
	\begin{aligned}
	\int_\alpha f = \int_\beta f
	\end{aligned}
	\end{equation}
\end{theorem}

\begin{theorem}[Regolarità]
	Se $f$ è olomorfa allora $f \in C^{\infty}(A)$ dove $A\subseteq \C$ è il dominio di $f$.
\end{theorem}

\begin{theorem}[Principio di identità]
	Sia $A\subset \C$ tale che $A \simeq \D$ per omotopie, allora se $f(x) = g(x)$ per ogni $x\in A$, allora $f = g$.
\end{theorem}
